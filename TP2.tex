\documentclass[11pt]{exam}
\RequirePackage{amssymb, amsfonts, amsmath, latexsym, verbatim, xspace, setspace}
\usepackage[french]{babel}
\usepackage[utf8]{inputenc}
\usepackage{url}
\usepackage[french]{babel}
\usepackage{graphicx}
\usepackage{hyperref}
\usepackage[]{geometry}
%\geometry {portrait, includehead, left=2cm, right=2cm, top=2cm, bottom=2cm}
%\hypersetup{
 %   unicode=false,          % non-Latin characters in Acrobat?s bookmarks
  %  colorlinks=true,       % false: boxed links; true: colored links
    %linkcolor=red,          % color of internal links (change box color with linkbordercolor)
    %citecolor=green,        % color of links to bibliography
    %filecolor=magenta,      % color of file links
    %urlcolor=magenta           % color of external links
%}

\pagestyle{headandfoot}

%\def\dbC{{\mathrm{I\hskip-4.7pt C}}}
%\def\dbR{{\mathrm{I\hskip-2.2pt R}}}
%\def\dbN{{\mathrm{I\hskip-2.2pt N}}}
%\def\esp{{\mathrm{I\hskip-1.5pt E}}}
%\def\pr{{\mathrm{I\hskip-2.2pt P}}}
%\def\clL{{\mathrm{\cal L}}}
%\def\cltL{{\mathrm{\widetilde{\cal L}}}}
%\def\w{{\mathbf{w}}}
%\def\x{{\mathbf{x}}}
%\def\Blambda{{\mathbf{\lambda}}} 

% =================================================================================
%\newcommand{\moncadre}[1]%
%{\fbox{
 %   \begin{minipage}{400pt} #1
  %  \end{minipage}
%}}

%\figgauche{LargeurImage}{FichierImage.eps}{TexteADroite}
%
\newlength\jataille
\newcommand{\figgauche}[3]%
{\jataille=1\linewidth\advance\jataille by -#1
      \advance\jataille by -.5cm
      \begin{minipage}[c]{6pc}
      \includegraphics[width=6pc]{#2}
      \end{minipage}\hfill
      \begin{minipage}[c]{30pc} #3
      \end{minipage}
		~\newline{}
}


% =================================================================================
\extraheadheight{50pt}
\extrafootheight{-70pt}
\firstpageheader{\includegraphics[width=100pt]{logoinsa} \\ L. Noiret, \\A. Rogozan \\ M.A. Haddouche}
{\Large \bf ~~TP \\ 
  Statistiques Descriptives \\ Analyse multi-dimensionnelle }
{\includegraphics[width=90pt]{logoasi} \\ $ 3^{\mbox{\footnotesize
      e}}$ ann\'ee}
\runningheader{ASI3}{Stats}{}
\headrule
\footer{}{}{p.\thepage/\numpages}

% For an exam, single spacing is most appropriate
% \singlespacing
% \onehalfspacing
% \doublespacing

% For an exam, we generally want to turn off paragraph indentation
\parindent 0ex
\newcommand{\matlab}[1]{ \verb|>>| #1 }

\begin{document} 
%\documentclass[11pt]{article}
%\usepackage{amsmath,amssymb}
%\usepackage{graphicx}
%\usepackage{changepage}
%\usepackage{rotating}
%\renewcommand{\baselinestretch}{1.2}
%\newcommand{\specialcell}[3][l]{%
 % \begin{tabular}[#1]{@{}c@{}@{}}#3\end{tabular}}
%\begin{document}
%\title{Analyse descriptive (suite)}

%\date{}
%\author{}
%\maketitle
 \section*{Quelques d\'efinitions}
\begin{itemize}
\item Le coefficient d'asym\'etrie (skewness en anglais) : mesure l'asym\'etrie d'une distribution d'une variable r\'eelle. Un coefficient positif (resp. n\'egatif) indique une queue de distribution \'etal\'ee vers la droite (resp. gauche).

%\item Le coefficient d'aplatissement (kurtosis): une mesure de l'aplatissement d'une courbe. Un coefficient d'aplatissement élevé indique que la distribution est plutôt pointue en sa moyenne, et a des queues de distribution épaisses 

\item Le coefficient de corr\'elation lin\'eaire (r): mesure l'intensit\'e de la liaison lin\'eaire entre deux variables quantitative. r varie entre -1 et 1. Une valeur proche de 0, indique que les donn\'ees ne sont pas li\'ees lin\'eairement, une valeur proche de -1 (resp. 1) indique que les donn\'ees sont fortement li\'ees n\'egativement (resp. postivement).

\item nuage de points (scatter plot en anglais) est une repr\'esentation de donn\'ees (x versus y) qui permet de mettre en \'evidence le type de liaisons entre deux variables.
\end{itemize}

 \section*{Description d'un base de donn\'ees cliniques}
On s'int\'eresse aux r\'esultats d'une \'etude clinique r\'ealis\'ee aux Etats-Unis. Le jeu de donn\'ees contient :
\begin{itemize}
\item $gender$ : le sexe du patient;
\item $ethnicity$ : l'origine \'ethnique d\'eclar\'ee (cette question est couramment pos\'ee  aux Etats Unis);
\item $age$ : l'\^age du patient au moment de l'\'etude;
\item $weight$ : le poids du patient ;
\item $protein$,   $protein2$,  $protein3$: la concentration de trois prot\'eines d'int\'er\^et dans le sang 
\item \textit{n\_visit} : le nombre de visites m\'edicales au cours des 24 derniers mois.
\end{itemize}

\section{Chargement et pr\'e-traitements des donn\'ees  } 
Importer le fichier $study.csv$ et enregistrer les donn\'ees au format table.\\
Quels sont les types des diff\'erentes variables? \\
\begin{itemize}
\item Les variables $gender$ et $ethnicity$ sont pour l'instant enregistr\'ees en tant que cha\^ines de caract\`eres. Nous allons les transformer en variables cat\'egorielles (ce qui nous facilitera leur analyse). Ex\'ecuter les lignes suivantes : \\
\indent \textit{study.ethnicity=categorical(study.ethnicity);}\\
\indent \textit{study.gender=categorical(study.gender);}
\item Supprimer les individus dont l'\^age est inf\'erieur \`a 15 ans :\\
\textit{study(study.age$\leq$15,:)=[];}\\
\end{itemize}



Utiliser la fonction $summary$ afin d'obtenir une vue d'ensemble rapide des donn\'ees. A premi\`ere vue, les donn\'ees contiennent-elles des valeurs ab\'errantes ou atypiques?

Tracer les boites a moustaches des variables quantitatives.

\section{Tableau des fr\'equences} 
Donner le tableau des fr\'equences $f$ et des fr\'equences cumul\'ees $F$ de la variable $n\_visit$ en adaptant le code suivant:\\
\textit{[n,edges]=histcounts(X,unique(X))\\
N=cumsum(n)}
\\
Etudier ces tableaux pour en d\'eduire la valeur de la m\'ediane, du 1er quartile et du 3e quartile.
\section{Repr\'esentation graphique unidimensionnelle}
Repr\'esenter graphiquement les effectifs des variables $gender$, $ethnicity$, $weight$.\\
%Quelle indicateur de tendance central pouvez-vous utiliser pour chaque variable?\\
%Utiliser la fonction $summary$ pour vous faire une premi\`ere id\'ee sur les donn\'ees.
Tracer les boites à moustaches des variables quantitatives.
\section{Repr\'esentation graphique multidimensionelle}
%Adapter le code suivant pour obtenir les nuages de points et histogrammes de toutes les variables quantitatives continues: \\
%\noindent \textit{dataLabels = \{`label 1',`label 2',`label 3'\}}\\
%$[H,AX] = plotmatrix(data);$\\
%\textit{for i = 1: length(AX)}\\
%\indent  \textit{ ylabel(AX(i,1),dataLabels\{i\});}\\
%\indent   \textit{ xlabel(AX(end,i),dataLabels\{i\});}\\
%\textit{end}\\
%Notes: la fonction $plotmatrix$ n'accepte que des donn\'ees de type array. Pour transformer une table en array, vous devez utiliser $table2array$.\\
%Pour r\'ecuper le noms des variables dans une table, utiliser : \textit{study.Properties.VariableNames}.
%\subsubsection{Coefficient d'asym\'etrie}
%Aide : la fonction \textit{`skewness'} de matlab prend en entr\'ee un array et non une table.    
Tracer les nuages de point de la variable $weigth$ en fonction des variables $protein,protein2$ et $protein3$\\
Comment d\'ecririez-vous les relations entre ces diff\'erentes variables ?\\
Calculer les coefficients de corr\'elations entre la variable $weigth$ et $protein,protein2$ et $protein3$. Commenter\\
Tracer les histogrammes des variables $age,protein, protein2$ et $weigth$.\\
A votre avis, les coefficients d'asym\'etrie  associ\'es aux variables $age$ et $protein2$ sont-ils positifs ou n\'egatifs? V\'erifier votre intuition en calculant le coefficient d'asym\'etrie sur toutes les variables quantitatives.\\

\section{Recodage d'une variable}
Ecrire un code pour recoder la variable $age$ en 5 classes d'effectifs \'egaux. Les r\'esulats seront stock\'es dans une variable de la table $study$ sous le nom  \textit{`age\_5cat'}.\\
Tracer l'histogramme de cette nouvelle variable.

Aide : regarder les fonctions $quantile$ et $discretize$.
\end{document}
